\documentclass[12pt]{article}
\usepackage{graphicx}

\begin{document}
    {\Huge \textbf{Viscous Drag} \par}

    \section*{Relevant formula}
    \begin{enumerate}
        \item $W = mg, \: weight \, = \, mass \times gravity$
        \item $\rho = \frac{m}{V}, \: density \, = \, \frac{mass}{volume}$
        \item $F = 6 \pi \eta rv, \: drag \, force \, = \, 6 \times \pi \times density \, 
        of \, fluid \times radius \times velocity$ 
    \end{enumerate}

    \section*{Conditions of Stokes' formula}
    \begin{enumerate}
        \item Small radius
        \item Small velocity
        \item Spherical ball
        \item Laminar flow
    \end{enumerate}

    \section*{Note}
    Viscosity changes with temperature. The higher the temperature the less viscous. The
    lower the temperature the higher the viscosity.

    The weight of fluid displaced by an object is the amount of upthrust it faces. So
    \[upthrust \, =  V\rho g, \: volume \, of \, object \times density \, of \, fluid \times gravity\]

    \section*{Derivation of big viscosity formula}
    We start with Stokes' law:
    \[F = 6\pi\eta rv\]
    If we draw the free body diagram then upthrust, $U$, is upwards. Weight, $W$, is downwards. Viscous 
    drag, $F$, is in the opposite direction of motion. Let's say the ball is traveling downwards so viscous
    drag is upwards. So we have $U + F = W$. Upthrust is same as weight of fluid displaced so, if $\rho_{1}$
    is the density of the ball and $\rho_{2}$ is the density of the fluid, we have:
    \[U = mass \, of \, fluid \, displaced \times g\]
    \[= V\rho_{2}g\] 
    \[W = mass \, of \, object \times g\]
    \[= V\rho_{1}g\]
    Subbing these into $U + F = W$ we get:
    \[V\rho_{2}g + 6\pi\eta rv = V\rho_{1}g\]
    \[6\pi\eta rv = Vg(\rho_{1} - \rho_{2})\]
    \[\eta = \frac{Vg(\rho_{1} - \rho_{2})}{6\pi rv}\]
\end{document}